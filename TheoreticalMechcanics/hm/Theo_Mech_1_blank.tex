\documentclass[UTF8,a4paper]{ctexart}
\usepackage{amsmath}
\usepackage{siunitx}
\usepackage{graphicx}
\usepackage{tikz}

\title{理论力学第一章作业}
\date{}
\author{}


\begin{document}
\maketitle
	
\paragraph{1.3} 曲线 $ OA = r $,以匀角速度 $ \omega $ 绕定点 $ O $ 转动.此曲柄借连杆 $ AB $ 使滑块 $ B $ 沿直线 $ Ox $ 运动.求连杆上 $ C $ 点的轨道方程及速度. 设 $ AC = CB = a, \angle AOB = \varphi , \angle ABO = \psi $.
	\begin{figure}[h]
		\centering
		\includegraphics[scale=0.8]{figs/1_3.jpg}
		\caption{1.3图}
	\end{figure}


\paragraph{1.7} 试自 \[ x = r \cos \theta , \quad y = r \sin \theta\] 出发,计算 $ \ddot{x} $ 及 $ \ddot{y} $ .并由此推出径向加速度 $ a_r $ 及横向加速度 $ a_\theta $.


\paragraph{1.9} 质点作平面运动,其速率保持为常数.试证其速度矢量 $ \boldsymbol{v} $ 与加速度矢量 $ \boldsymbol{a} $ 正交.

	
\paragraph{1.11} 质点沿着半径为 $ r $ 的圆周运动,其加速度矢量与速度矢量间的夹角 $ \alpha $ 保持不变. 求质点的速度随时间变化的规律.已知初速度为 $ v_0 $.


\paragraph{1.15} 当一轮船在雨中航行时,它的雨篷遮着篷的垂直投影后 \SI{2}{\metre} 的甲板,篷高 \SI{4}{\meter}. 但当轮船停航时,甲板上干湿两部分的分界线却在篷前 \SI{3}{\metre}. 如果雨点的速率为 \SI{8}{\meter \per \second}, 求轮船的速率.


\paragraph{1.19} 将质量为 $ m $ 的质点竖直向上抛入有阻力的介质中.设阻力与速度平方成正比, 即 $ R = m k^2 g v^2 $.如上掷时的速度为 $ v_0 $, 试证此质点又落至投掷点时的速度为\[ v_1 = \frac{ v_0 }{ \sqrt{1+k^2 {v_0}^2}} \]

\paragraph{1.27} 一质点自一水平放置的光滑固定圆柱面凸面的最高点自由滑下.问滑至何处,此质点将离开圆柱面?假定圆柱体的半径为 $ r $.

\paragraph{1.33} 光滑钢丝圆圈的半径为 $ r $, 其平面为竖直的. 圆圈上套一小环,其重为 $ W $. 如钢丝圈以匀加速度 $ a $,沿竖直方向运动,求小环的相对速度 $ v_r $ 及圈对小环的反作用力 $ R $.

\paragraph{1.37} 根据湯川核力理论,中子与质子之间的引力具有如下形式的势能: \[ V\left( r \right) = \frac{ k \mathrm{e}^{-ar}}{r} \quad \left( k \leq 0 \right)  \]


\paragraph{1.45} 如 $ \dot{s}_a $ 及 $ \dot{s}_p $ 为质点在远日点及近日点处的速率, 试证明\[ \dot{s}_p : \dot{s}_a = (1+e) : (1-e) \]

\end{document}
